%%\chapter{Revisão da literatura}
\label{Fundamentacao}
%
%	\begin{flushright}
%		\textit{``O período de maior ganho em conhecimento e experiência
%		\\ é o período mais difícil da vida''.\\Dalai Lama}
%	\end{flushright}
%
%

Neste capítulo busca-se contextualizar o leitor dos fundamentos teóricos nos quais este trabalho se baseia, mais especificamente abordando de forma detalhada o funcionamento da técnica DTM. Além disso, as tecnologias utilizadas na realização desta pesquisa serão apresentadas de forma teórica, permitindo ao leitor assimilar seu funcionamento e assim melhor compreender sua aplicação prática descrita no capítulo \ref{Desenvolvimento}.

Para melhor compreensão, este capítulo é dividido em quatro seções. Na seção \ref{Fundamentacao:DTM} ocorre a demonstração de como funciona a técnica de memorização dinâmica de traces de forma teórica. Na seção \ref{Fundamentacao:DTMHardware} vemos uma adaptação da técnica para aplicação prática em hardware. Na seção \ref{Fundamentacao:LEON3} é apresentado o processador LEON3 juntamente com outras tecnologias auxiliares. Por fim, na seção \ref{Fundamentacao:FPGA} é explicado como funciona o FPGA, tecnologia para gravação de circuitos utilizada neste trabalho.


\section{Memorização dinâmica de traces}
\label{Fundamentacao:DTM}

A memorização dinâmica de traces é uma técnica que busca evitar a perda de tempo computacional reduzindo o número de execuções de instruções já executadas. Diferente de algumas técnicas de programação que exploram o reuso de computação a nível de software, a DTM trabalha a nível de hardware e identifica candidatos ao reuso de forma dinâmica. Essa identificação dinâmica se caracteriza por reutilizar instruções em qualquer ponto do programa, categorizando uma instrução ou sequência de instruções como redundante de acordo com os valores dos seus parâmetros de entrada, comparando-os aos de uma execução anterior \cite{costa2001explorando}.

Na figura \ref{Fig:ExemploDTM} vemos um exemplo do funcionamento de DTM em um fluxo de controle. Na figura, cada nó representa um instrução e as ligações entre os nós representam os possíveis caminhos de execução de um código. 

Em (a) é possível observar uma sequência de intruções que foram executadas pelo programa, com os nós em cinza claro simbolizando as instruções executadas enquanto os em branco os caminhos não tomados.

Em (b) temos em cinza escuro os nós que representam instruções redundantes executadas. Após a identificação dessas instruções, ocorre a construção do trace e a memorização deste.

Em (c) é representado como o fluxo de controle se comporta quando ocorre o reuso do trace. Como pode ser notado, as instruções identificadas como redundantes não são executadas. Ocorre então a escrita dos resultados gerados na execução memorizada e o desvio para a próxima instrução a ser executada que não pertence ao trace, considerando qualquer desvio que tenha sido tomado entre as instruções reusadas.

\begin{figure}
\label{Fig:ExemploDTM}
	\caption[Exemplo das etapas do processo de DTM em um fluxo de controle]{
	Exemplo do processo de DTM em um fluxo de controle: (a) Instruções executadas; (b) Instruções redundantes identificadas, memorização; (c) Fluxo de controle quando ocorre reuso do trace.}
	
	\centering
	\begin{multicols}{3}
		\begin{tikzpicture}[->,>=stealth,shorten >=1pt,auto,node distance=1cm,thick,main node/.style={fill=black!20,circle,draw,font=\sffamily}]
	
		\node[main node] (1) { };
		\node[main node] (2) [below of=1] { };
		\node[main node] (3) [below of=2] { };
		\node[main node, style={thin, fill=none}] (4) [below left of=3] { };
		\node[main node] (5) [below right of=3] { };
		\node[main node, style={draw=none, fill=none}] (erapraser6) [below left of=4] { };
		\node[main node] (6) [below left of=5] { };
		\node[main node, style={thin, fill=none}] (7) [below right of=5] { };
		\node[main node, style={thin, fill=none}] (8) [below right of=6] { };
		\node[main node] (9) [below left of=6] { };
		\node[main node, style={draw=none, fill=none}] (erapraser10) [below of=7] { };
		\node[main node, style={draw=none, fill=none}] (10) [below of=8] { };
		\node[main node] (11) [below of=9] { };
		\node[main node] (12) [below right of=11] { };
		\node[main node, style={thin, fill=none}] (13) [below left of=11] { };
		\node[main node] (14) [below of=12] { };
		\node[main node, style={draw=none, fill=none}] (15) [below of=13] { };
		\node[main node] (16) [below of=14] { };
		\node[main node] (17) [below left of=16] { };
		\node[main node, style={thin, fill=none}] (18) [below right of=16] { };
		\node[main node] (19) [below of=17] { };
		\node[main node, style={draw=none, fill=none}] (20) [below of=16] {};
		\node[main node, style={draw=none, fill=none}] (21) [below of=20] {};
		\node[main node, style={draw=none, fill=none}] (a) [below of=21] {(a)};
		
		\path[every node/.style={font=\sffamily\small}]
		(1) edge (2)
		(2) edge (3)
		(3) edge (4)
			edge (5)
		(4)	edge (erapraser6)
		(5) edge (6)
			edge (7)
		(6) edge (8)
			edge (9)
		(7) edge (erapraser10)
		(8) edge (10)
		(9) edge (11)
		(11)edge (12)
			edge (13)
		(12)edge (14)
		(13)edge (15)
		(14)edge (16)
		(16)edge (17)
			edge (18)
		(17)edge (19);
		
		\end{tikzpicture}
		
		%%%%%%%%%%%%%%%%%%%%%%%%%%%%%%%%%%%%%%%%%%%%%%%%%%%%%%%%%%%%%%%%%%
		

		\begin{tikzpicture}[->,>=stealth,shorten >=1pt,auto,node distance=1cm,thick,main node/.style={fill=black!50,circle,draw,font=\sffamily}]
		
		\node[main node, style={fill=black!20}] (1) { };
		\node[main node] (2) [below of=1] { };
		\node[main node] (3) [below of=2] { };
		\node[main node, style={thin, fill=none}] (4) [below left of=3] { };
		\node[main node] (5) [below right of=3] { };
		\node[main node, style={draw=none, fill=none}] (erapraser6) [below left of=4] { };
		\node[main node] (6) [below left of=5] { };
		\node[main node, style={thin, fill=none}] (7) [below right of=5] { };
		\node[main node, style={thin, fill=none}] (8) [below right of=6] { };
		\node[main node] (9) [below left of=6] { };
		\node[main node, style={draw=none, fill=none}] (erapraser10) [below of=7] { };
		\node[main node, style={draw=none, fill=none}] (10) [below of=8] { };
		\node[main node] (11) [below of=9] { };
		\node[main node] (12) [below right of=11] { };
		\node[main node, style={thin, fill=none}] (13) [below left of=11] { };
		\node[main node] (14) [below of=12] { };
		\node[main node, style={draw=none, fill=none}] (15) [below of=13] { };
		\node[main node] (16) [below of=14] { };
		\node[main node] (17) [below left of=16] { };
		\node[main node, style={thin, fill=none}] (18) [below right of=16] { };
		\node[main node, style={fill=black!20}] (19) [below of=17] { };
		\node[main node, style={draw=none, fill=none}] (20) [below of=16] {};
		\node[main node, style={draw=none, fill=none}] (21) [below of=20] {};
		\node[main node, style={draw=none, fill=none}] (b) [below of=21] {(b)};
		
		\path[every node/.style={font=\sffamily\small}]
		(1) edge (2)
		(2) edge (3)
		(3) edge (4)
		edge (5)
		(4)	edge (erapraser6)
		(5) edge (6)
		edge (7)
		(6) edge (8)
		edge (9)
		(7) edge (erapraser10)
		(8) edge (10)
		(9) edge (11)
		(11)edge (12)
		edge (13)
		(12)edge (14)
		(13)edge (15)
		(14)edge (16)
		(16)edge (17)
		edge (18)
		(17)edge (19);
		
		\end{tikzpicture}
		
		%%%%%%%%%%%%%%%%%%%%%%%%%%%%%%%%%%%%%%%%%%%%%%%%%%%%%%%%%%%%%%%%%%
		
		
		\begin{tikzpicture}[->,>=stealth,shorten >=1pt,auto,node distance=1cm,thick,main node/.style={fill=black!50,circle,draw,font=\sffamily}]
		
		\node[main node, style={fill=black!20}] (1) { };
		\node[main node] (2) [below of=1] { };
		\node[main node] (3) [below of=2] { };
		\node[main node, style={thin, fill=none}] (4) [below left of=3] { };
		\node[main node] (5) [below right of=3] { };
		\node[main node, style={draw=none, fill=none}] (erapraser6) [below left of=4] { };
		\node[main node] (6) [below left of=5] { };
		\node[main node, style={thin, fill=none}] (7) [below right of=5] { };
		\node[main node, style={thin, fill=none}] (8) [below right of=6] { };
		\node[main node] (9) [below left of=6] { };
		\node[main node, style={draw=none, fill=none}] (erapraser10) [below of=7] { };
		\node[main node, style={draw=none, fill=none}] (10) [below of=8] { };
		\node[main node] (11) [below of=9] { };
		\node[main node] (12) [below right of=11] { };
		\node[main node, style={thin, fill=none}] (13) [below left of=11] { };
		\node[main node] (14) [below of=12] { };
		\node[main node, style={draw=none, fill=none}] (15) [below of=13] { };
		\node[main node] (16) [below of=14] { };
		\node[main node] (17) [below left of=16] { };
		\node[main node, style={thin, fill=none}] (18) [below right of=16] { };
		\node[main node, style={fill=black!20}] (19) [below of=17] { };
		\node[main node, style={draw=none, fill=none}] (20) [below of=16] {};
		\node[main node, style={draw=none, fill=none}] (21) [below of=20] {};
		\node[main node, style={draw=none, fill=none}] (c) [below of=21] {(c)};
		
		\path[every node/.style={font=\sffamily\small}]
		(1) edge (2)
		%(2) edge (3)
		(3) edge (4)
		edge (5)
		(4)	edge (erapraser6)
		(5) edge (6)
		edge (7)
		(6) edge (8)
		edge (9)
		(7) edge (erapraser10)
		(8) edge (10)
		(9) edge (11)
		(11)edge (12)
		edge (13)
		(12)edge (14)
		(13)edge (15)
		(14)edge (16)
		(16)edge (17)
		edge (18)
		(17)edge (19);
		
	    \draw 
	    (2.west)
	    -- ++(left:4.5em) 
	    |- (19.west);
		
		\end{tikzpicture}
	\end{multicols}

\legend{Fonte: elaborada pelo autor}
\end{figure}

\subsection{Identificação e memorização de traces}
\label{Fundamentacao:DTM:Identificacao}

A primeira etapa do processo de DTM é a identificação de traces. Traces são conjuntos sequenciais de instruções válidas redundantes e que não possuem efeitos colaterais. Instruções são consideradas redundantes quando, dado um determinado conjunto de entradas, a instrução produzirá sempre o mesmo resultado. Exemplos de instruções redundantes e sem efeitos colaterais incluem instruções lógicas e aritméticas e de desvio condicional e incondicional. Instruções que manipulam a memória primária e que executam sub-rotinas do sistema não são redundantes e possuem efeitos colaterais, portanto não serão constituintes de um trace. No caso da arquitetura SPARC também foram consideradas não redundantes as instruções que manipulam as janelas de registradores, já que adicionaram uma complexidade muito grande à criação e armazenamento de traces.

Operações envolvendo ponto-flutuante possuem a mesma caracterização quanto à redundância e causalidade de efeitos colaterais que as instruções citadas acima. Por exemplo, instruções aritméticas com ponto-flutuante são redundantes enquanto a manipulação de memória envolvendo-os não. Porém, como \citeonline{gabbay1996speculative} observa, instruções de ponto flutuante costumam ter baixa localidade espacial, o que no caso da construção de uma unidade de DTM as torna indesejáveis, por adicionarem complexidade desnecessária tendo em vista o baixo retorno causado por sua inclusão. Isso não impede que a técnica de memorização seja empregada em operações de ponto-flutuante para aumentar seu desempenho, como demonstrado por \citeonline{citron1998accelerating}.

Após a identificação do trace, se dá o processo de memorização. A memorização consiste em armazenar as informações necessárias para permitir a identificação de uma oportunidade de reuso de um trace, além de garantir que o circuito seja capaz de gerar corretamente as saídas necessárias e realizar um desvio para a próxima instrução a ser executada.



\subsection{Reuso de traces}
\label{Fundamentacao:DTM:Reuso}

O reuso de um trace ocorre quando este é redundante e possui uma instância memorizada equivalente ao que deve ser executado. Essas instâncias são equivalentes quando possuem a mesma sequência de instruções e o mesmo contexto de entrada.

O contexto de entrada é definido como o conjunto de valores utilizados no trace cuja origem é externa ao trace. De forma análoga, o contexto de saída são os valores gerados internamente ao trace e que estão disponíveis para uso externo ao término deste.

Como exemplo, tomemos um trace para o pseudocódigo demonstrado na figura \ref{Fig:ExemploContexto1}. Em um trace gerado após a execução desse código, os valores contidos em $a$, $b$ e $c$ são utilizados internamente antes de terem seus valores definidos no próprio trace, compondo então o contexto de entrada.

\begin{figure}[!h]
	\label{Fig:ExemploContexto1}
	\caption[Exemplo de pseudocódigo]{
		Exemplo de pseudocódigo.}
	
		\begin{algorithmic}
			\STATE $c \leftarrow c + a$
			\STATE $x \leftarrow c + b$
			\IF{$x \le 5$}
			\STATE $y \leftarrow x * 2$
			\ELSE
			\STATE $x \leftarrow x + 1$
			\ENDIF
		\end{algorithmic}
	\legend{Fonte: elaborada pelo autor}
\end{figure}

Suponhamos os valores de entrada como $a = 1$, $b = 2$ e $c = 3$. Com esses valores as instruções armazenadas no trace resultante se assemelhariam com o trace representado na figura \ref{Fig:ExemploContexto2}\footnote{Estariam inclusas no trace também as instruções de comparação e desvio responsáveis pelo desvio condicional, mas por terem formato e efeitos relativos à arquitetura foram omitidas para maior clareza no exemplo.}. 
%. Estariam inclusas no trace também as instruções de comparação e desvio responsáveis pelo desvio condicional, mas por terem formato e efeitos relativos à arquitetura foram omitidas para maior clareza no exemplo.

\begin{figure}[!h]
	\label{Fig:ExemploContexto2}
	\caption[Exemplo de trace para o pseudocódigo da figura \ref{Fig:ExemploContexto1}]{
		Exemplo de trace para o pseudocódigo da figura \ref{Fig:ExemploContexto1}.}
	
	\begin{algorithmic}
		\STATE $c \leftarrow c + a$
		\STATE $x \leftarrow c + b$
		\STATE $x \leftarrow x + 1$
	\end{algorithmic}
	\legend{Fonte: elaborada pelo autor}
\end{figure}

Com os valores do contexto de entrada contidos no parágrafo anterior, o contexto de saída deste trace é $c = 4$ e $x = 7$. Assim, sempre que os valores do contexto de entrada se igualassem a esses ao entrar no trace, não há necessidade de execução das instruções, ocorrendo então a escrita do contexto de saída diretamente. Como essas instruções não possuem efeito colateral, esse processo é transparente para o programa sendo executado.

É possível então observar a importância da memorização correta do trace. Ainda analisando o pseudocódigo da figura \ref{Fig:ExemploContexto1}, caso o contexto de entrada seja $a = 1$, $b = 2$ e $c = 1$, o contexto de saída é $c = 2$, $x = 4$ e $y = 8$. O trace gerado para essa execução difere do trace descrito anteriormente, tanto nas instruções contidas como nos resultados gerados, de forma que o reuso do trace impróprio pode levar a aplicação que está executando a produzir resultados incorretos.

\section{DTM em hardware}
\label{Fundamentacao:DTMHardware}

Na seção \ref{Fundamentacao:DTM} é explanado o funcionamento da técnica DTM de forma abstrata, sem detalhar como implementar em hardware um mecanismo capaz de executar tal tarefa. Esta seção apresenta uma possível implementação, descrita por \citeonline{costa2001explorando}.

A implementação de \citeonline{costa2001explorando} tem como alvo um processador com ISA MIPS I, uma ISA do tipo RISC tal qual a ISA SPARC, se assemelhando em muitas características ao processador LEON3 utilizado neste trabalho. Assim, a implementação apresentada nesta seção é adaptada para se adequar à ISA e desenho do LEON3.

Para armazenamento das informações relevantes ao reuso de um trace será utilizada uma unidade de memória denominada \tablet. Essa unidade de memória será organizada como uma tabela, na qual cada linha corresponde a um trace armazenado. Também é utilizada uma tabela para o armazenamento de informações sobre instruções individuais, denominada \tableg. Esta será utilizada para o reuso de instruções isoladas, enquanto aquela para o reuso de blocos de instruções sequenciais.

\subsection{A unidade \tableg}
\label{Fundamentacao:DTMHardware:TableG}

Como descrito anteriormente, o primeiro passo do processo de DTM é a identificação de instruções redundantes. Para que o reuso dessas instruções possa ser feito é necessário que as instâncias executadas sejam armazenadas em alguma estrutura, de forma que quando for identificada a redundância o resultado prévio possa ser prontamente utilizado.

Para cumprir este papel, foi projetada uma unidade de memória denominada \tableg. A \tableg\ tem como função armazenar instâncias anteriores e permitir a leitura de resultados. Para que esses valores possam ser lidos e gravados de forma eficiente, essa tabela foi projetada tendo em cada linha uma instância de uma instrução redundante e em cada coluna um campo que deve ser armazenado.

\begin{figure}
	\label{Fig:MemoTableG}
	\caption[Representação da tabela \tableg]{
		Representação da tabela \tableg.}
	\begin{center}
		%A famosa gambiarra ataca novamente
		\newcommand{\tabela}[1]{
			\multicolumn{1}{|c|}{$#1$}
		}
		\begin{tabular}{*{8}{c}}
			tamanho em bits & $32$ & $32$ & $32$ & $32$ & $1$ & $1$ & $1$ \\
			\cline{2-8}
			& \tabela{pc} & \tabela{sv1} & \tabela{sv2} & \tabela{res/targ} & \tabela{jmp} & \tabela{brc} & \tabela{btaken} \\
			\cline{2-8}
			& \multicolumn{7}{c}{\vdots} \\
			\cline{2-8}
			& \tabela{pc} & \tabela{sv1} & \tabela{sv2} & \tabela{res/targ} & \tabela{jmp} & \tabela{brc} & \tabela{btaken} \\
			\cline{2-8}
		\end{tabular}
	\end{center}
	\legend{Fonte: elaborada pelo autor}
\end{figure}

Na figura \ref{Fig:MemoTableG} pode ser vista a distribuição dos bits na \tableg. Abaixo é definido o significado dos valores a serem estocados em cada campo da tabela:

\begin{itemize}
	\item $pc$: Responsável por armazenar o valor do contador de programa quando a instrução foi executada. Esse valor nada mais é que o endereço de memória onde está localizada a instrução.
	\item $sv1$: Armazena o valor do primeiro parâmetro passado para a instrução.
	\item $sv2$: Armazena o valor do segundo parâmetro passado para a instrução.
	\item $res/targ$: Campo onde é salvo o resultado da computação realizada, seja o resultado de uma instrução lógica ou aritmética ou o endereço para um desvio.
	\item $jmp$: Caso a instrução seja um desvio incondicional será setado em $1$. Caso contrário estará em $0$.
	\item $brc$: Caso a instrução seja um desvio condicional será setado em $1$. Caso contrário estará em $0$.
	\item $btaken$: Caso a instrução seja um desvio condicional e tenha sido tomado será setado em $1$. Caso contrário não tenha sido tomado estará em $0$. Caso não seja um desvio condicional seu valor é indeterminado.
\end{itemize}

Colocando esses valores armazenados em \tableg\ no contexto da técnica DTM, o campo $pc$ é utilizado para identificar a qual instrução pertence aquela instância armazenada. Como é necessário o armazenamento do campo $pc$ para identificação, não há razão para armazenar a tabela de outra forma que não completamente associativa. Os valores de $sv1$ e $sv2$ compõe o contexto de entrada de uma única instrução, podendo esta fazer uso de apenas um ou ambos de acordo com seu formato. Em $res/targ$ temos o contexto de saída da instrução.

Os três campos de um bit servem para identificar como deve ser utilizado o resultado. Caso os três tenham valor $0$, $res/targ$ é copiado para o registrador de destino indicado na instrução. Caso $jmp$ seja $1$, o valor de $res/targ$ é somado ao registrador de programa, o PC, realizando assim um desvio incondicional. Caso $brc$ esteja ativo a instrução é um desvio condicional, e o valor de $res/targ$ é somado ao PC caso $btaken$ também esteja ativo. Independentemente do valor de $btaken$, caso $brc$ esteja ativo ambos os bits serão utilizados para atualizar a unidade de predição de desvios, caso este exista.

\subsection{A unidade \tablet}
\label{Fundamentacao:DTMHardware:TableT}

Analogamente à \tableg, a \tablet\ tem como objetivo armazenar dados de instâncias de instruções armazenadas afim de permitir o reuso destas em execuções futuras. Porém, diferentemente daquela, esta armazena informações sobre traces, sendo então responsável por guardar todas as informações necessárias para o reuso de um conjunto de duas ou mais instruções redundantes.

O projeto da \tablet, apesar de compartilhar características com a \tableg, deve ser adaptado para que o armazenamento de informações sobre um trace completo possam ser utilizadas de forma a identificar um trace redundante e reutilizá-lo de forma transparente à aplicação utilizando a unidade de processamento. 

A \tablet\ é então também uma unidade de memória completamente associativa organizada em forma tabular na qual cada linha representa um trace, ou seja uma instância de execução de uma sequência de instruções, e cada coluna um campo necessário para identificação ou reuso correto deste trace. A descrição destes campos, que podem ser vistos na figura \ref{Fig:MemoTableT}, segue abaixo:

\begin{itemize}
	\item $pc$: Onde é guardado o endereço de memória da primeira instrução do trace, usado para identificação de candidatos a reuso.
	\item $npc$: Armazena o endereço de memória da próxima instrução a ser executada. Após o reuso de um trace é necessário um desvio para a instrução subsequente, sendo então utilizado o valor de $npc$ para o cálculo desse desvio.
	\item $icr$: Identifica quais registradores pertencem ao contexto de entrada.
	\item $icv$: Armazena os valores do contexto de entrada, contido nos registradores indicados pelo campo $icr$.
	\item $ocr$: Identifica quais registradores pertencem ao contexto de saída.
	\item $ocv$: Armazena os valores do contexto de saída, contido nos registradores indicados pelo campo $ocr$.
	\item $bmask$: Máscara na qual cada bit em nível alto indica a presença de um desvio no trace.
	\item $btaken$: Máscara que armazena para cada desvio indicado em $bmask$ se ele foi tomado ou não.
\end{itemize}

\begin{figure}
	\label{Fig:MemoTableT}
	\caption[Representação da tabela \tablet]{
		Representação da tabela \tablet.}
	\begin{center}
		%A famosa gambiarra ataca novamente novamente
		\newcommand{\tabela}[1]{
			\multicolumn{1}{|@{ }c@{ }|}{#1}
		}
		
		\newcommand{\tabelatripla}[2]{
			\tabela{$#1_{1}$} & \tabela{\hdots} & \tabela{$#1_{#2}$}
		}
		
		\tiny
		\begin{tabular}{*{21}{c}}
			tamanho em bits & $32$ & $32$ & \multicolumn{3}{c}{$5 * N_{1}$} & \multicolumn{3}{c}{$32 * N_{1}$} & \multicolumn{3}{c}{$5 * N_{2}$} & \multicolumn{3}{c}{$32 * N_{2}$} & \multicolumn{3}{c}{$1 * B$} & \multicolumn{3}{c}{$1 * B$} \\
			\cline{2-21}
			& \tabela{$pc$} & \tabela{$npc$} & \tabelatripla{icr}{N_{1}} & \tabelatripla{icv}{N_{1}} & \tabelatripla{ocr}{N_{2}} & \tabelatripla{ocv}{N_{2}} & \tabelatripla{bmask}{B} & \tabelatripla{btaken}{B}  \\
			\cline{2-21}
			\multicolumn{21}{c}{\vdots} \\
			\cline{2-21}
			& \tabela{$pc$} & \tabela{$npc$} & \tabelatripla{icr}{N_{1}} & \tabelatripla{icv}{N_{1}} & \tabelatripla{ocr}{N_{2}} & \tabelatripla{ocv}{N_{2}} & \tabelatripla{bmask}{B} & \tabelatripla{btaken}{B}  \\
			\cline{2-21}
		\end{tabular}
		\normalsize
		
	\end{center}
	\legend{Fonte: elaborada pelo autor}
\end{figure}

A figura \ref{Fig:MemoTableT} também indica o uso de três parâmetros configuráveis, $N_{1}$, $N_{2}$ e $B$. 

$N_{1}$ define quantos registradores podem pertencer ao contexto de entrada de um trace. Caso o para continuar a construção do trace esse seja necessário mais registradores, a construção será encerrada. 

Com uso bastante semelhante a $N_{1}$, $N_{2}$ limita a quantidade de registradores no contexto de saída. \citeonline{costa2001explorando} utiliza $N_{1} = N_{2}$, mas isso não é necessário para o funcionamento correto do DTM. Em aplicações que os traces comumente possuem mais valores no contexto de saída ou de entrada pode-se ajustar a unidade para melhor se adaptar as características dos traces nela armazenados. 

Por fim, $B$ indica a quantidade máxima de instruções de desvio que serão armazenadas em um trace. Os valores armazenados em $bmask$ e $btaken$ são utilizados para atualizar as unidades de predição de desvio. Caso a aplicação envolva muitas instruções de desvio, aumentar o valor de $B$ aumentará o tamanho máximo dos traces armazenados, permitindo o reuso de mais instruções com uma única entrada.

Esses parâmetros devem ser definidos no momento da construção do hardware. A existência desses se deve ao fato de não interferirem no bom funcionamento da técnica mas influenciar os resultados. A medida que se aumenta o valor dos parâmetros o tamanho máximo dos traces também aumenta, o que pode causar um ganho de desempenho dependendo da aplicação sendo executada. Porém, esses valores são diretamente proporcionais à quantidade de memória necessária para o armazenamento, aumentando a área de chip e o custo do hardware resultante.


\subsection{Construção e reuso}
\label{Fundamentacao:DTMHardware:Integracao}

Para construção de um trace são utilizados dois mapas de contexto e um \textit{buffer} temporário. Esse \textit{buffer} possui o mesmo formato de uma linha da \tablet, enquanto os mapas possuem um bit para cada registrador, totalizando 32 bits.

Quando uma instrução redundante é identificada se inicia a construção do trace. O valor de $pc$ é inserido no \textit{buffer}, enquanto seu valor somado 4 no $npc$. Para cada registrador pertencente ao contexto de entrada, o respectivo bit na máscara de contexto de entrada é posta em nível alto. Caso o bit estivesse em $0$, o número do registrador e seu valor são inseridos no \textit{buffer}. O mesmo se aplica ao mapa de contexto de saída, com a diferença que caso haja duas escritas em um registrador o valor em $ocv$ é atualizado, o que não ocorre no contexto de entrada. Caso a instrução a ser executada seja um desvio, os devidos bits são escritos nas máscaras do \textit{buffer} temporário. 

Para cada instrução subsequente, caso seja redundante, o campo $npc$ recebe o valor do registrador PC mais quatro, enquanto os outros valores são inseridos e atualizados como descrito anteriormente. Caso a instrução não seja redundante o \textit{buffer} é escrito em uma entrada da \tablet\ e os valores do \textit{buffer} e dos mapas são zerados.

Paralelamente, para cada instrução é verificado em \tableg\ e \tablet\ se existem entradas para as quais valor de $pc$ equivale ao do registrador PC. Para as que são encontradas, é verificado então o contexto de entrada. Os registradores indicados pelo campo $icr$ tem seus valores lidos do banco de registradores e comparados com os valores de $icv$ para cada entrada identificadas de \tablet. Já para \tableg, os registradores indicados na própria instrução tem seus valores comparados com os campos $sv1$ e $sv2$. Caso haja um trace em \tablet\ no qual todos os registradores passem neste teste, ele então é reusado. Caso não haja um trace mas haja um instrução em \tableg\, ela então é reusada. Caso contrário, a instrução é executada normalmente.

\section{O LEON3}
\label{Fundamentacao:LEON3}

O processador LEON3 faz parte do GRLIB, um conjunto de projetos de propriedade intelectual para desenvolvimento de \textit{system-on-chip} mantidos pela Cobham Gaisler. Escrito em VHDL e altamente configurável, é possível sintetizar e gravar os elementos da GRLIB em diversas plataformas e editar e simulá-las nas diversas ferramentas de CAD utilizadas no desenvolvimento de hardware em HDL.

