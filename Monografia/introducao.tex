
%\chapter{Introdução}

	\begin{flushright}
		\textit{``Clássico é clássico e vice-versa''.\\
		Mário Jardel}
	\end{flushright}


%\textbf{Organização de informação não se dá da mesma forma em qualquer domínio.}
%As metodologias e técnicas tradicionais de organização de informação (OI) permitem que informação seja coletada, indexada, classificada e recuperada, na Web ou em outros repositórios, tendo em vista seu consequente uso por uma comunidade de usuários. No entanto, é comum a necessidade de se implementar métodos com foco em repositórios e conteúdos de áreas específicas, por aqueles métodos tradicionais não suportarem o reconhecimento de termos de todos os domínios ou não atenderem a necessidades muito específicas dos usuários de informação. Esse é o caso do tratamento de fórmulas químicas \cite{sun07}, de citações bibliográficas \cite{giles98} e de referências geográficas \cite{overellPhd09}, assim como de repositórios de patentes \cite{patents11} e de informação corporativa \cite{hu2010enterprisecase}.
%
%%\textbf{SRIC ainda costumam ser SRI sem qualquer adaptação especial.}
%Porém, ainda é comum que sistemas de recuperação de informação corporativa sejam apenas sistemas tradicionais de recuperação de informação sobre repositórios de documentos corporativos. 
%Dessa forma, apenas pequena parcela da tecnologia é adaptada para atender as necessidades das empresas e de seus usuários. 
%Isso frequentemente leva a insatisfação de usuários de informação e causa prejuízos aos processos organizacionais que dependam de buscas \cite{solomon2002}.
%
%%\textbf{O usuário costuma precisar de vários SRI para realizar a mesma tarefa.}
%Adicionalmente, as tarefas de usuários corporativos normalmente exigem informação de diferentes fontes de informação \cite{halevy2005enterprise}, tornando comum o acesso a tais fontes por meio de diferentes sistemas de recuperação de informação. 
%Apesar de complexo, o acesso através de uma única e integrada interface é muito importante para qualquer usuário de informação, além de recomendável para garantir uma gestão eficaz de recursos de informação.
%
%%\textbf{Tudo isso é culpa da complexidade da informação corporativa. Conhecê-la melhor é a solução.}
%Essas dificuldades são comuns em decorrência da complexidade naturalmente encontrada dentro de organizações, onde diferentes profissionais, atividades e unidades organizacionais são empregados para realizar trabalho. A recuperação mais eficaz de informação corporativa e o desenvolvimento de metabuscas mais eficientes parecem depender da compatibilização e do controle de características que permeiem a informação da empresa, o que sintetiza a justificativa desta pesquisa.

A Lei de Moore possui algumas variações quanto ao seu enunciado, porém todas afirmam que a capacidade computacional dos processadores cresceria exponencialmente devido aos avanços na tecnologia. Por 50 anos essa previsão se manteve consistente com os produtos lançados no mercado, como descrito em. Porém, limitações físicas na criação de circuitos integrados ameaçam o continuidade dessa evolução. \cite{50yearsmooreslaw}

Mas com o crescente aumento da demanda por computação é necessário que os projetistas encontrem maneiras de aperfeiçoar ainda mais o funcionamento das unidades de processamento. Uma solução que vem sendo utilizada é acoplar vários processadores para funcionar em paralelo, porém isso aumenta a complexidade de projetos tanto a nível de hardware como de software, além de amplificar o consumo energético do sistema.

O grande desafio da arquitetura de computadores é buscar soluções eficientes, conciliando fatores como desempenho do sistema, consumo de energia, custo de produção e tamanho e complexidade do produto final. Em muitas situações, esses fatores concorrem entre si, levando o projetista a ter de tomar decisões sobre qual abordagem será escolhida para solucionar determinado problema.

O que ocorre então é a criação de sistemas especialistas para determinadas funções, enquanto outros projetos mais gerais lidam com uma gama mais diversa de aplicações. Em ambos os casos, projetistas consideram qual problema buscam resolver para criar a solução mais adequada dentro das restrições. 

Como exemplo podemos comparar as diferentes abordagens assumidas ao projetar um \textit{system-on-chip} para aplicação em um sistema embarcado e na criação de uma unidade de processamento gráfico. Enquanto sistemas embarcados prezam por tamanho reduzido e baixo consumo de energia, unidades gráficas têm como prioridade a velocidade para cálculos de ponto flutuante, sendo otimizadas para executar instruções simples a diversos dados de entrada simultaneamente. \cite{tanenbaumsco}




\section{Justificativa} %Inseri para organizar o texto

%\textbf{Necessidade de SRI adequados para informação corporativa, que evoluam continuamente e que busquem a generalidade.}
Esta pesquisa justifica-se pela necessidade de se implementar sistemas automáticos de recuperação de informação que deem suporte adequado à informação corporativa e às tarefas dos usuários corporativos, mas também que favoreçam a provisoriedade da informação e a interoperabilidade entre sistemas. Para isso, deve-se buscar uma compreensão mais geral da informação corporativa, que beneficie sua evolução contínua e não a limite a um cenário de uso excessivamente reduzido.

%\textbf{Há diversas iniciativas para aumentar o desempenho dos componentes dos SRI.}
Há diversas iniciativas que tentam aumentar o desempenho dos componentes de sistemas de recuperação de informação corporativa e dos processos que os suportam, da coleta de documentos à interface com o usuário. As iniciativas mais comuns dependem de vocabulário controlado \cite{ontology12}, o que nem sempre é simples de gerir dentro de empresas. A adoção de vocabulário controlado enfrenta um grande volume de informação produzida, a dificuldade de avaliar a conformidade de novos documentos, e ainda a realidade de que nem toda empresa conta com profissionais de informação qualificados, sendo que suas funções eventualmente são assumidas pelos próprios usuários corporativos. Adicionalmente, nem todo usuário de informação possui habilidades suficientes para executar atividades de produção, classificação, indexação e busca de informação da forma mais eficiente.
Outras iniciativas dependem de relacionamentos entre documentos em forma hipertextual, assim como ocorre na \textit{Web}. Esse não é um recurso muito comum em coleções corporativas, onde memorandos, notas fiscais, manuais, folhas de pagamento e balanços são conectados apenas através de códigos textuais, nem sempre fáceis de se reconhecer automaticamente.

%\textbf{Maioria das iniciativas dão-se sobre a perspectiva dos projetistas ou sobre a perspectiva dos usuários, o que gera dois problemas: ou requer um projeto prévio intenso ou perde em flexibilidade para evolução.}
No entanto, a maioria dessas iniciativas continua a ser fortemente dependente da perspectiva dos projetistas dos sistemas de recuperação de informação, com reflexos provenientes da forma como a informação é estruturada nos diversos repositórios usados pela empresa; ou da perspectiva dos usuários, com reflexos provenientes da forma como o conhecimento é organizado. Embora natural, essa dependência exige um grande esforço prévio de projeto ou então provoca uma excessiva perda de flexibilidade para a evolução do sistema de recuperação de informação. 

%\textbf{Estruturas classificatórias facetadas resolvem os dois problemas anteriores.}
Para contornar essas limitações, estruturas classificatórias facetadas têm sido adotadas como base para um modelo mais flexível e incremental de desenvolvimento de sistemas de recuperação de informação \cite{broughton2006,docubrowse10}. Adicionalmente, uma vez que documentos apresentam facetas significativas para o contexto do documento e do usuário da informação, normalmente consultas submetidas aos mecanismos de busca também incluem estas facetas, como nomes de pessoas e de instituições, localidades e datas \cite{rosieJones08,docubrowse10,mir2ed}. Tal contextualização, além de favorecer a recuperação de informação, pode contribuir para melhorar o \textit{ranking} da informação e para criar novos serviços e funcionalidades \cite{borges07}.

%A origem da classificação facetada se deu com estudos de \citeonline{ranganathan1967} e teve continuidade nos estudos de outros autores da área de Biblioteconomia e Ciência da Informação tais como \citeonline{vickery2008faceted} e \citeonline{broughton2006}, apesar do conceito estar em uso crescente também em outras áreas com algumas variações \cite{broughton2006}. As facetas identificadas a partir desta pesquisa constituem características comuns da informação corporativa de duas empresas, mas favorecem a análise de um domínio corporativo na medida em que o método pode ser repetido em organizações adicionais, usando documentos de diferentes tipos, propósitos e idiomas. 

%Na teoria da análise facetada de \citeonline{ranganathan1967}, faceta é um conceito central, embora apresente interpretações variadas na literatura \cite{broughton2006}. No entanto, faceta sempre é vista como uma característica permanente e relevante de uma entidade, algumas vezes reconhecida por meio da análise facetada. Análise facetada é uma técnica através da qual facetas são descobertas em um domínio e normalmente formam uma classificação facetada \cite{labarre2010}. A abordagem facetada constitui uma estratégia classificatória pela qual se procura responder como uma entidade pode ser descrita, ao invés de procurar responder onde a entidade deve ser posicionada dentro de uma estrutura classificatória definida \textit{a priori} \cite{ranganathan1967}. Assim, variadas perspectivas e incertezas sobre necessidades de usuários são consideradas, dentro de um processo analítico, incremental e iterativo, normalmente tornando seu produto útil para mais de um problema.

%\textbf{Facetas significativas são descobertas por uma análise de domínio.}
As facetas mais significativas do domínio corporativo podem ser descobertas através de um processo de análise de domínio. E por meio de facetas significativas, a organização da informação pode ser aperfeiçoada e sistemas de recuperação de informação mais eficientes são possíveis \cite{albrechtsen1993subject}. Porém, os requisitos para a análise de todo um domínio e para a análise da informação de uma única empresa são diferentes \cite{gopinath92}. Enquanto o domínio requer uma compreensão mais geral e consensual, a informação de uma instituição admite perspectivas baseadas em maior empirismo e pragmatismo, sendo útil apenas em um contexto mais limitado \cite{hjorland2002domain}.

%\textbf{Análise de domínio deve ser cuidadosa para reconhecer as características essenciais.}
As facetas normalmente restringem-se às mais importantes por razões econômicas e práticas mesmo havendo muitas características potencialmente relevantes ao domínio \cite{ranganathan1967}. É possível descobrir características do domínio através de técnicas como da análise facetada. No entanto, distinguir entre características importantes e essenciais só se dá através de um rigoroso processo de análise de domínio \cite{lykke2011domain}.

%\textbf{Características locais também são importantes para a análise de domínio.}
Por outro lado, as características locais da informação corporativa, a terminologia específica de uma atividade econômica, as necessidades individuais de cada unidade organizacional e o contexto geográfico e histórico da empresa também são importantes no processo de análise de domínio. Além disso, elas são úteis para descrever a informação corporativa, inclusive com uma precisão que supera aquela possibilitada por características mais gerais e genéricas. Também, as características locais são representadas da mesma forma que aquelas mais gerais do domínio \cite{dolby09extractingVocabularies,giess2008generation,wild2009describing}, mesmo não sendo compartilhadas por todas as empresas. No entanto, sua adoção implica em uma modelagem que se afasta de um modelo genérico de empresa.

%\textbf{Informação implícita também é importante.}
Em todo sistema de recuperação de informação corporativa, os usuários da informação estão interessados tanto em dados explícitos presentes no conteúdo de documentos \cite{gardin1973} quanto em informação implícita e conhecimento informal distribuído nos sistemas corporativos \cite{dealwis2001singapore,choo2008,marcella2012,nunes2006,ofarrill2010}, o que constitui um desafio adicional para a representação do domínio. Portanto, o domínio, no nível mais global ou mais individual, é dinâmico e sua representação deve estar preparada para atualizações contínuas.

%\textbf{Análise facetada se justifica como técnica para análise de domínio.}
Exatamente por favorecer um processo incremental de análise de domínio, a análise facetada parece apropriada. Porém, não há relatos da aplicação dessa técnica em documentos corporativos na literatura. Desenvolvida sobre uma base teórica sólida a partir da Biblioteconomia e da Ciência da Informação \cite{garfield1984}, a análise facetada é mais popular na sua área de origem e para a classificação bibliográfica, embora esteja se popularizando em outras áreas \cite{labarre2010}. São exemplos não exaustivos os documentos de engenharia \cite{giess2008generation,wild2009describing}, \textit{sites} corporativos \cite{wang2008using}, documentos \textit{Web} \cite{hong06,vickery2008faceted} e catálogos \textit{Web} \cite{sacco2006}. Como a diversidade de aplicações e ramos do conhecimento tem aumentado, surgem também visões alternativas e um grande número de trabalhos sem que todos os princípios originais de \citeonline{ranganathan1967} sejam respeitados \cite{spiteri98simplified,wild2009describing}.

%A análise facetada da informação corporativa, como uma técnica de análise de um domínio, deve começar por definir as facetas mais comuns e frequentes à maioria das empresas. As sugestões mais tradicionais partem das cinco categorias consideradas fundamentais por \citeonline{ranganathan1967}, personalidade, matéria, energia, espaço e tempo, e as treze recomendadas pelo \textit{Classification Research Group}: coisa, tipo, parte, propriedade, material, processo, operação, agente, paciente, produto, subproduto, espaço e tempo. Elas supostamente têm o potencial de representar o conhecimento de qualquer domínio, sendo usadas como ponto de partida para a análise facetada \cite{broughton2006}.





%\textbf{Conclusão da justificativa.}
Dessa maneira, a organização da informação por meio de estruturas classificatórias facetadas pode contribuir positivamente para a melhoria dos processos de classificação e indexação de documentos, e para outras etapas da construção e uso dos sistemas de recuperação de informação. Essa hipótese se baseia na presença de facetas comuns a documentos corporativos e na flexibilidade desse método de organização.
% Uma segunda hipótese refere-se ao relacionamento entre facetas, dentro de documentos ou mesmo dentro da coleção, como importante recurso para reconhecimento e desambiguação de termos e entidades de forma automática, o que por sua vez suportaria uma indexação automática eficiente de entidades facetadas. %Aqui pode haver um exemplo e/ou uma figura de exemplo.







%\section{Justificativa}

%Por outro lado, muito pouco se sabe da eficiência dessas técnicas de recuperação quando adotadas em conjunto, e.g. assunto, tempo e espaço \citepx{hassan09}. %Parte desse desconhecimento deve-se ao volume menor de implementações de SRI baseados em estruturas facetadas \citepx{hong06} e também ao custo de classificar grandes coleções por meio de análise facetada.

\section{Problema}

%\textbf{As questões de pesquisa.}
As seguintes questões constituem o problema desta pesquisa: 
A informação corporativa possui características que potencialmente sejam comuns a todo o domínio corporativo? 
No contexto de uso de sistemas de recuperação de informação corporativa, as expressões de busca de usuários apresentam as mesmas características que estão presentes nos documentos? 
A organização facetada da informação corporativa contribui para o aumento do desempenho geral do sistema automático de recuperação de informação?

%\textbf{Conceituação dos termos novos presentes nas perguntas acima.}
As características da informação, em documentos e expressões de busca, referem-se a atributos que devem ser reconhecidos a partir do domínio corporativo. Adicionalmente, a contribuição esperada da organização facetada da informação está associada à sua capacidade de representar entidades por meio da totalidade ou de um subconjunto de suas características, suportando a recuperação de entidades e consequentemente a recuperação de documentos onde as entidades estão contidas. Finalmente, o desempenho do sistema de recuperação de informação refere-se à sua capacidade atender às necessidades informacionais de seus usuários, algo que pode ser parcialmente medido através de métricas discutidas na literatura.

%Finalmente, a contribuição esperada da organização facetada da informação aos sistemas de recuperação de informação está associada à sua capacidade de identificar entidades por meio da totalidade ou subconjunto de seus atributos, emprestando a documentos e outras entidades parte de seu contexto.


%automaticamente termos a facetas
%aumento da precisão como medida de desempenho do SRI


%A organização multifacetada da informação demanda um esforço manual de identificação de facetas e anotação de termos da coleção ou exige um processo de reconhecimento que seja automático e suficientemente preciso para torná-la útil na recuperação de documentos.
%que caracterizam o sistema de informação corporativo pode ajudar no reconhecimento automático e na desambiguação de entidades multifacetadas e seus valores de facetas (focus)

\section{Objetivos}

%\textbf{Objetivo geral.}
Para responder ao problema proposto, o objetivo geral deste trabalho é propor um conjunto de características da informação corporativa que favoreça a organização e a recuperação da informação.
% projetar e descrever um processo de reconhecimento e desambiguação de termos que permita uma associação dinâmica entre cada termo reconhecido e sua respectiva faceta, tais como nomes de pessoas e de instituições, processos, localidades e datas.

%associação dinâmica. Como um processo automático, termo pode ser associado a uma faceta diferente daquela com a qual já se encontra associado a partir da reunião de novas evidências, pela edição de um documento ou pela simples inclusão de outros documentos no índice. Assim, a técnica deve ser aplicada recursivamente e deve ser considerada eficaz caso o número de associação falso-positiva seja baixo.

%\textbf{Objetivos específicos.}
Também, objetivam-se mais especificamente:

\begin{enumerate}
	\item Propor um conjunto de facetas que seja útil na organização automática da informação corporativa e na interoperabilidade entre diferentes repositórios de informação da mesma empresa ou de diferentes empresas;

	\item Identificar as facetas pelas quais os usuários de informação especificam sua necessidade de informação no contexto de trabalho;

	\item Avaliar as implicações da organização facetada da informação corporativa no desempenho dos sistemas automáticos de recuperação de informação.

\end{enumerate}

%\textbf{Link para os procedimentos metodológicos.}
Para responder as questões propostas e aos objetivos apresentados, pretende-se empreender uma análise preliminar do domínio corporativo a partir de dois exemplares do domínio, avaliar como os usuários informacionais mobilizam as facetas mais comuns do domínio, e implementar e avaliar um protótipo de sistema de recuperação de informação corporativa. Os procedimentos metodológicos seguidos são apresentados no capítulo \ref{metodologia}.













%\section{Contribuições esperadas}

%Pelo desenvolvimento desta pesquisa em organização de informação facetada, espera-se que sejam realizados aperfeiçoamentos em técnicas usuais de i) reconhecimento e desambiguação de termos, ii) busca de documentos por meio de consultas ambíguas, e iii) ordenação de resultados a partir de indicadores facetados.

%Na atividade de reconhecimento, o reconhecimento de termos permite a contextualização de documentos coletados para um sistema de busca, o que, em outros trabalhos, tem sido feito em classes homogêneas de documentos. Esta pesquisa pretende que documentos mais diversos, comuns em empresas, como manuais, memorandos, bancos de dados estruturados ou páginas Web sejam tratados sob o mesmo arcabouço, considerando as diferenças de gênero linguístico e ao mesmo tempo em que se possa adequadamente estendê-lo a outros idiomas. Para esse fim, deve-se implementar uma técnica de desambiguação que utilize evidências de documentos explicitamente ou implicitamente relacionados, com ou sem hipertexto, e que avalie automaticamente a qualidade da fonte de evidência de facetas.

%Para isso, deve ser implementado um protótipo funcional de sistema de recuperação de informação corporativo, com componentes de coleta, \textit{parsing}, indexação, processamento de consulta e \textit{ranking}. Tais componentes, modificados para reconhecer, desambiguar, indexar e recuperar informação multifacetada a partir dos documentos da coleção, servirão para avaliar o modelo de organização multifacetada de informação em outros cenários de teste.

%Para a atividade de busca pelo usuário, uma interface deve ser projetada para favorecer o ajuste fino dos critérios de busca, textuais e multifacetados. O método de interação com as respostas da busca pode representar um passo adiante na ordenação de respostas. Adicionalmente, um algoritmo de ordenação de documentos deve ser proposto para o mesmo arcabouço, pelo \textit{ranking} a partir dos critérios textuais e multifacetados, mas com alterações na estrutura de indexação que favoreçam o método de \textit{ranking}. Espera-se obter um algoritmo de busca eficiente por informação multifacetada, especialmente no contexto de sistemas de recuperação de informação corporativos. A avaliação da modelagem multifacetada para a organização da informação corporativa também é uma contribuição esperada.

%É necessário investigar porque as estruturas próprias não têm produzido aumento da eficiência na recuperação das respostas \cite{anastacio09}, e se isso é também verdade em buscas onde exista o ajuste fino do \textit{ranking}. É prevista a necessidade de se projetar uma estrutura de dados mais apropriada para indexar informação corporativa e multifacetada. A estrutura de dados deve privilegiar operações de atualização de índice (inclusão, alteração e exclusão de documentos da coleção) e de acesso de dados para recuperação e \textit{ranking}.

%Finalmente, os repositórios usados para avaliação de técnicas de recuperação de informação corporativa têm sido constituídos por classes similares de documentos. Logo, espera-se formar um repositório de teste e avaliação a partir de documentos de diferentes gêneros linguísticos e formatos, o que pode contribuir positivamente para pesquisas futuras. Para isso, este trabalho deve primeiramente colaborar para o projeto, criação e disponibilização de uma coleção de referência em língua portuguesa, adequado para o contexto empresarial, de fonte pública, apropriado para avaliação de SRI corporativos. A nova coleção de referência não será a única base de experimentação deste trabalho, mas sim complementar àquelas disponíveis na literatura.

%As contribuições desta pesquisa mostram-se particularmente úteis aos sistemas de RI corporativos, em qualquer idioma, mas podem igualmente ser aplicados em outros domínios. Dessa forma, não podem ser descartadas as possibilidades de se gerar produtos úteis e funcionais no curto prazo.







%\section{Resultados esperados}

%A técnica de consulta multifacetada deve favorecer a precisão e a revocação se comparada às técnicas estatísticas como TF-IDF, nosso baseline, com custo computacional conhecido.

%A técnica implementada e o método proposto devem ser mais eficientes que os baselines taxonomia facetada, TF-IDF, taxonomia facetada dinâmica, busca sobre metadados na biblioteca digital implementada em dspace.

%O desempenho também deve ser superior na coleção de referência da TREC comparada com resultados disponíveis na literatura.









\section{Estrutura da tese}

%\textbf{Links para os próximos capítulos.}
Esta tese está estruturada em seis capítulos, ordenados pelo momento em que foram concluídos dentro do ciclo de vida desta pesquisa, e anexos, a saber:

\begin{itemize}

	%\item capítulo atual	

	\item O capítulo \ref{metodologia} apresenta os procedimentos metodológicos através dos quais este trabalho se desenvolve, o que inclui a formação das coleções de documentos para experimentação e avaliação, as etapas para o projeto do protótipo funcional, bem como as estratégias de validação.

	\item As bases teóricas são apresentadas em seguida, no capítulo \ref{literatura}, o que inclui marcos conceituais importantes para o contexto de sistemas de recuperação de informação corporativa e para a organização do conhecimento e da informação em estruturas facetadas.

	\item Uma análise preliminar de domínio é descrita no capítulo \ref{analiseDominio}. Duas coleções de documentos são investigadas para identificar as facetas mais comuns que possam constituir um conjunto mínimo de facetas para representar entidades no domínio corporativo.

	\item No capítulo \ref{prototipo} é detalhada uma avaliação da coleção pública e é obtido um conjunto de expressões de busca a partir de seus usuários sobre uma das coleções estudadas. São então avaliadas a utilidade e a eficiência de um modelo de recuperação baseado em facetas do domínio corporativo.

	\item Finalmente, no capítulo \ref{conclusao} são apresentadas as conclusões e considerações finais, principais contribuições, limitações e indicadas algumas direções para trabalhos futuros.

	\item A tese inclui parte dos seus resultados e produtos em anexos, tendo em vista que sua extensão poderia comprometer a legibilidade do texto e a compreensão do leitor. As coleções estudadas não são disponibilizadas entre os anexos pois isso violaria alguns direitos de propriedade intelectual dos seus autores. A tese aponta outros meios de obter uma cópia das referidas coleções.
	
\end{itemize}

