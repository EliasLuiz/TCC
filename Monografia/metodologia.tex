%\chapter{Procedimentos metodológicos}
\label{Metodologia}

%	\begin{flushright}
%		\textit{``Um bom começo é a metade''.\\
%		Aristóteles}
%	\end{flushright}


%\textbf{Introdução das etapas metodológicas.}
Está pesquisa é descritiva em se tratando dos objetivos, já que tem como objetivo descrever as características observadas do objeto de pesquisa. É aplicada do ponto de vista de sua natureza, considerando que busca contrastar os resultados obtidos em modelos teóricos e simulações com resultados medidos em uma implementação real. É classificada como qualitativa quanto à abordagem ao problema, pois os resultados apresentados são medições de grandezas do objeto de estudo.

Os procedimentos metodológicos adotados podem ser descritas na seguinte sequência:

\begin{enumerate}

\item realizar um levantamento sobre arquiteturas de computadores disponíveis em código aberto, considerando sua compatibilidade com os equipamentos disponíveis no laboratório e características da arquitetura e implementação, selecionando a que mais se adequasse a um dos padrões propostos;

\item reunir e estudar trabalhos publicados relacionados ao problema principal abordado, que em suma são os trabalhos onde a DTM é primeiramente apresentada e alguns outros trabalhos relacionados analisando algumas facetas diferentes da mesma;

\item adaptar a técnica DTM para esse conjunto de instruções de máquina (\textit{Instruction Set Architecture - ISA}), implementar a unidade de memorização dinâmica de traces em linguagem de descrição de hardware, testá-la de forma isolada e integrá-la ao processador da maneira menos invasiva, isto é alterando ao mínimo a unidade de processamento; 

\item realizar testes em um processador sem DTM, que age como unidade de controle, e no processador com DTM, comparando os resultados de ambos obtidos através de  simulação e execução física em placas de FPGA (\textit{Field-programmable gate array});

\end{enumerate}

Essas etapas serão expostas em mais detalhes nas seções subsequentes deste capítulo.

\section{Escolha da Arquitetura} 
\label{Metodologia:Arquitetura}

O primeiro passo do desenvolvimento deste trabalho foi selecionar uma arquitetura de processadores que atendesse os requisitos necessários, listados abaixo:

\begin{itemize}
	
	\item Disponível em código aberto na forma de alguma linguagem de descrição de hardware, preferencialmente em Verilog ou VHDL;
	
	\item Sintetizável e gravável no FPGA disponibilizado pela instituição, o chip Altera Cyclone II EP2C35F672C6;
	
	\item ISA do tipo RISC (\textit{Reduced Instruction Set Computer}) ou Java;
	
	%%%%%%%%%%%%%%%%%%%%%%%%%%%%%%%%%%%%%%%%%%%%%%%%%%%%%%%%%%%%%%%%
	%%%%%%%%%%%%%%%%%%%%%%%%%%%%%%%%%%%%%%%%%%%%%%%%%%%%%%%%%%%%%%%%
	%%%%%%%%%%%%%%%%%%%%%%%%%%%%%%%%%%%%%%%%%%%%%%%%%%%%%%%%%%%%%%%%
	%%%%%%%%%%%%%%%%% ADICIONAR MAIS COISAS AQUI %%%%%%%%%%%%%%%%%%%
	%%%%%%%%%%%%%%%%%%%%%%%%%%%%%%%%%%%%%%%%%%%%%%%%%%%%%%%%%%%%%%%%
	%%%%%%%%%%%%%%%%%%%%%%%%%%%%%%%%%%%%%%%%%%%%%%%%%%%%%%%%%%%%%%%%
	%%%%%%%%%%%%%%%%%%%%%%%%%%%%%%%%%%%%%%%%%%%%%%%%%%%%%%%%%%%%%%%%
	
\end{itemize}

O motivo para que a escolha da arquitetura se desse antes da revisão de literatura se dá pelo fato desta ser influenciada pelo tipo da arquitetura utilizado. Enquanto grande parte dos trabalhos utilizados seriam os mesmos, a bibliografia básica descrevendo a DTM está sujeita ao tipo da arquitetura.

Caso fosse selecionada uma arquitetura baseada em Java como o JOP \cite{schoeberl2005jop}, em que o funcionamento do processador é baseado em uma máquina de pilha, seria utilizado como referência básica \cite{silva2006memorizacao}, já que este trabalho lida de maneira mais detalhada com a técnica DTM aplicada a uma máquina Java.

Porém, por possuir melhor compatibilidade com o FPGA utilizado, decidiu-se utilizar a arquitetura LEON3, atualmente mantido pela Aeroflex Gaisler, que possui como ISA o SPARC v8 \cite{gaisler2010leon3}. Por ser uma arquitetura de modelo RISC tal qual a arquitetura MIPS, utilizada em \cite{costa2001explorando}, este foi tido como a peça de bibliografia central para este trabalho.


\section{Revisão da literatura}
\label{Metodologia:Literatura}

Como dito anteriormente na seção \ref{Metodologia:Arquitetura}, a peça central da literatura utilizada neste trabalho é \cite{costa2001explorando} devido a sua abordagem minuciosa ao descrever todas as características que compõe a técnica de memorização dinâmica de traces. Sendo o trabalho mais completo tratando de DTM e contendo uma explicação detalhada em como implementar essa técnica em um processador RISC, é notável a semelhança com este trabalho, o que o torna uma referência de suma importância.

Cabe notar que, enquanto há outros trabalhos que trazem diferentes abordagens sobre DTM, como reuso especulativo através da predição de valores de entrada, neste trabalho o escopo foi limitado a uma abordagem mais simples, analisando o impacto dos conceitos básicos de DTM em um processador. Outras abordagens adicionando uma maior capacidade e complexidade à unidade serão tratadas na seção \ref{Conclusao:Trabalhos}, onde serão apresentadas algumas possíveis adições para futuros trabalhos que compartilhem essa temática.


%%%%%%%%%%%%%%%%%%%%%%%%%%%%%%%%%%%%%%%%%%%%%%%%%%%%%%%%%%%%%%%%
%%%%%%%%%%%%%%%%% ESTICAR EM MAIS PARAGRAFOS? %%%%%%%%%%%%%%%%%%
%%%%%%%%%%%%%%%%%%%%%%%%%%%%%%%%%%%%%%%%%%%%%%%%%%%%%%%%%%%%%%%%

Porém, devido à natureza prática deste trabalho, a revisão de literatura não se ateve ao estudo teórico da técnica. Uma seção considerável da bibliografia utilizada concerne às tecnologias componentes do trabalho, desde manuais confeccionados pela Altera tratando aspectos físicos do chip de FPGA utilizado à documentos da Aeroflex Gaisler referentes à comunicação com o processador para \textit{debug} de software. Essas referências utilizadas na implementação serão melhor demonstradas nos capítulos \ref{Literatura} e \ref{Desenvolvimento}, onde ocorrerá um maior detalhamento das questões técnicas envolvidas no projeto.


\section{Adaptação e Implementação}
\label{Metodologia:Implementacao}


\section{Testes e Análise}
\label{Metodologia:Analise}

Após completada a implementação do processador com DTM, será iniciada a etapa de testes e análise dos resultados. 

%%%%%%%%%%%%%%%%%%%%%%%%%%%%%%%%%%%%%%%%%%%%%%%%%%%%%%%%%%%%%%%%
%%%%%%%%%%%%%%%%%%%%%%%%%%%%%%%%%%%%%%%%%%%%%%%%%%%%%%%%%%%%%%%%
%%%%%%%%%%%%%%%%%%%%%%%%%%%%%%%%%%%%%%%%%%%%%%%%%%%%%%%%%%%%%%%%
%%%%%%%%%%%%%%%%% ESPECIFICAR QUAL BENCHMARK %%%%%%%%%%%%%%%%%%%
%%%%%%%%%%%%%%%%%%%%%%%%%%%%%%%%%%%%%%%%%%%%%%%%%%%%%%%%%%%%%%%%
%%%%%%%%%%%%%%%%%%%%%%%%%%%%%%%%%%%%%%%%%%%%%%%%%%%%%%%%%%%%%%%%
%%%%%%%%%%%%%%%%%%%%%%%%%%%%%%%%%%%%%%%%%%%%%%%%%%%%%%%%%%%%%%%%

Como descrito em \ref{Introducao:Objetivos}, serão realizados testes para determinar a área de chip, potência dissipada, latência de ciclo de execução e ganho de performance na execução de programas de \textit{benchmark}.

Para que os resultados tenham valor significativo, será utilizado o mesmo processador sem DTM como controle, servindo de referência para os valores lidos e permitindo uma visão mais completa dos impactos do mecanismo nas características gerais da unidade de processamento.

%%%%%%%%%%%%%%%%%%%%%%%%%%%%%%%%%%%%%%%%%%%%%%%%%%%%%%%%%%%%%%%%
%%%%%%%%%%%%%%%%%%%%%%%%%%%%%%%%%%%%%%%%%%%%%%%%%%%%%%%%%%%%%%%%
%%%%%%%%%%%%%%%%%%%%%%%%%%%%%%%%%%%%%%%%%%%%%%%%%%%%%%%%%%%%%%%%
%%%%%%%%%%%%% FAZER CITACOES (CHU, PAT E TANEM?) %%%%%%%%%%%%%%%
%%%%%%%%%%%%%%%%%%%%%%%%%%%%%%%%%%%%%%%%%%%%%%%%%%%%%%%%%%%%%%%%
%%%%%%%%%%%%%%%%%%%%%%%%%%%%%%%%%%%%%%%%%%%%%%%%%%%%%%%%%%%%%%%%
%%%%%%%%%%%%%%%%%%%%%%%%%%%%%%%%%%%%%%%%%%%%%%%%%%%%%%%%%%%%%%%%

A área do chip será determinada pela quantidade de células ou \textit{slices} necessários para a gravação do circuito em FPGA. Essa métrica impacta diretamente no tamanho do circuito resultante, custo por unidade produzida usando tecnologias ASIC (\textit{Application Specific Integrated Circuits}) e qual o número mínimo de células que um dispositivo lógico programável deve possuir para suportá-lo. 

Além disso, a área do chip impacta indiretamente a potência dissipada e a latência do circuito. Indiretamente pois, apesar de não haver relação de causalidade direta, comumente existe uma correlação entre circuitos com uma área de chip maior e maior quantidade de elementos lógicos, o que pode acarretar em uma perda de potência maior.

A potência dissipada no circuito, ou seja, a energia consumida por este, é outro aspecto importante a ser considerado ao determinar a aplicabilidade da técnica e que pode ser medido diretamente com auxilio de equipamentos para medições elétricas. Além de possuir uma relação direta com o calor produzido no circuito, explicada pelo efeito Joule, a fonte utilizada para alimentação do circuito deve possuir a capacidade de suprir a potência necessária. Isso é um fator determinante em aplicações de sistemas embarcados, nos quais muitas vezes busca-se o sistema com o menor consumo possível devido a estar embutido em outros sistemas, normalmente em um ambiente não idealmente preparado e estar ativado por longos períodos de tempo.

Também é possível perceber uma correlação entre área de chip e latência, apesar de mais fraca. O ponto que envolve ambos é o chamada caminho crítico, isto é, o maior caminho de dados possível dentro do circuito. Este caminho é o fator determinante para a determinação da latência do circuito e está diretamente associado com a quantidade de elementos pelos quais o sinal deve passar no circuito. Como cada elemento contido no circuito possui um \textit{delay} ou atraso próprio, a quantidade de elementos do caminho crítico é diretamente proporcional ao atraso total do caminho, sendo este o fator que determina a latência do circuito. Portanto, a relação entre área de chip e latência do circuito depende do design deste, já que um projeto que explore características de paralelismo é capaz de possuir uma área maior sem necessariamente estender o caminho crítico.

A latência por sua vez define qual o tempo de ciclo ou \textit{clock} mínimo para o circuito sem que haja perda de dados, portanto determinando o limite máximo para a frequência de trabalho do processador. Apesar de haverem outros fatores envolvidos no desempenho final do circuito, o tempo de ciclo é um fator fundamental. Com a implementação de DTM, será observado como foi alterado o caminho crítico e portanto como foi alterado o tempo mínimo de \textit{clock}, medindo qual a frequência máxima possível para o circuito.

Juntamente com isto serão executados programas de \textit{benchmark} para determinar ganhos em tempo de execução de diferentes tarefas, cada programa executando um tipo de aplicação e quantificando o desempenho do sistema. Assim, juntamente com a informação sobre alteração na frequência máxima suportada, será possível ter uma visão mais completa sobre os impactos que a implantação de DTM em uma unidade de processamento possui no desempenho final do sistema.




%%\textbf{2 conjuntos da revisão: trabalhos relacionados; bases de comparação da avaliação experimental}
%A revisão de literatura empregada neste trabalho resulta em dois conjuntos distintos de trabalhos relacionados: i) a revisão do estado da arte em análise de domínio e recuperação de informação; e ii) o levantamento bibliográfico dos principais resultados de trabalhos anteriores que explicitaram características do domínio corporativo, com origem em diversas áreas do conhecimento.%, encerrado após a conclusão da análise preliminar de domínio.
%
%%\textbf{Onde encontrar o primeiro conjunto.}
%O estado da arte e da técnica em análise de domínio e recuperação da informação é apresentado no capítulo \ref{literatura}. O contexto corporativo foi adotado para limitar o escopo deste trabalho e torná-lo viável para estudos com usuários e para classificação intelectual de documentos. Com isso, trabalhos relacionados mais especificamente à informação corporativa são especialmente de interesse. Na literatura sobre informação corporativa, também se busca um conjunto de facetas para a informação corporativa, como se observa na figura \ref{fig:metodologia}. Características da informação corporativa naturalmente permeiam os produtos tecnológicos de recuperação de informação corporativa e são discutidas em trabalhos sobre o tema nas áreas de Ciência da Computação e Biblioteconomia e Ciência da Informação. Entretanto, a fragmentação desse conhecimento torna difícil produzir um mapa das características do domínio corporativo. Adicionalmente, como os trabalhos tendem a assumir uma perspectiva mais pragmática, as características corporativas já conhecidas tendem a ser úteis e restritas apenas ao seu contexto original de estudo.
%
%%\textbf{Onde encontrar o segundo conjunto.}
%O segundo produto da revisão engloba um conjunto com os principais métodos automáticos e semiautomáticos de classificação, indexação e \textit{ranking} de informação, bem como as métricas usadas para avaliação experimental do desempenho dos métodos. Devidamente estudados, devem servir de base de comparação para projetar métodos mais adequados para organizar e recuperar informação corporativa. O segundo produto é apresentado na seção \ref{prototipo-colecaoPublica}, precisamente onde é adotado para validar os resultados deste trabalho sobre a coleção pública.
%
%%Tais métodos foram implementados dentro do \textit{software} de gerenciamento de bibliotecas digitais como parte da prototipação.

%\section{Coleções de documentos}
%
%%\textbf{2 coleções.}
%Neste trabalho, duas coleções de documentos corporativos são adotadas para a análise de domínio. A primeira coleção refere-se à coleção de referência usada na trilha \textit{Enterprise} da \textit{Text Retrieval Conference} (TREC) até o ano de 2008. Trata-se de uma coleção de 370.715 páginas \textit{Web} públicas da \textit{Commonwealth Scientific and Industrial Research Organisation} (CSIRO). A segunda coleção refere-se a um conjunto mais amplo de documentos, o que inclui atas, relatórios, memorandos, e-mails e páginas \textit{Web} públicas.
%
%%\textbf{uma coleção é pública.}
%A coleção de referência da \textit{Text Retrieval Conference} é útil por facilitar a comparação entre experimentos empíricos relatados na literatura e experimentos empreendidos nesta tese. Porém, seu uso tem sido criticado por contar exclusivamente com páginas \textit{Web} pública da CSIRO, criando condições que em nada se parecem com aquelas presentes em um ambiente real de busca corporativa. %A caracterização desta coleção é apresentada na seção 4.1.%Considerações de outros autores sobre seu uso e sobre metodologias de avaliação serão apresentadas na seção \ref{avaliacao}.
%
%%\textbf{uma coleção é particular.}
%Na tentativa de reduzir as limitações da coleção de referência, é adotada uma segunda coleção de documentos. Trata-se de um conjunto de documentos de uma empresa pública brasileira, com uma diversidade maior de tipos de documentos que aquela encontrada na coleção de referência. Ela se caracteriza como um ambiente mais próximo da realidade de uma empresa e das necessidades de informação de um usuário corporativo real, porém não pode ser vista como uma substituta da coleção de referência da \textit{Text Retrieval Conference} e nem mesmo como uma coleção perfeita para todo e qualquer objetivo. Nesta tese, a segunda coleção é denominada como coleção particular. No entanto, a coleção também se tornará publicamente disponível no ano de 2015. %A caracterização da nova coleção é apresentada na seção 4.2.%A segundo coleção de documentos foi concluída antes mesmo que a revisão da literatura fosse encerrada. 
%
%%\textbf{Classificação de tarefas de buscas Versus Presença de usuários reais.}
%Documentos da coleção pública são classificados para diferentes tarefas de busca. As tarefas de busca são aquelas mais comumente realizadas por usuários de informação reais, responsáveis pela classificação intelectual dos documentos para a avaliação, sendo que os usuários não são conhecidos. Ao contrário, para estudos sobre a coleção particular está disponível uma amostra de usuários reais.
%
%%\textbf{2 experimentos.}
%A presença de duas coleções provoca impactos em vários procedimentos metodológicos, como a figura \ref{fig:metodologia} ilustra, requerendo dois experimentos e avaliações diferentes. Enquanto a coleção particular faz uso de seus usuários para validar os resultados deste trabalho, a coleção pública conta com resultados prévios obtidos em duas edições da \textit{Text Retrieval Conference}, em 2007 e 2008.
%
%%\textbf{Validação das coleções.}
%Finalmente, a coleção particular tem sido usada por seus usuários para a realização de trabalho e para a tomada de decisão nos últimos anos. Portanto, partindo do pressuposto que a coleção particular seja uma coleção corporativa válida, pretende-se validar também a coleção pública como uma coleção corporativa ao demonstrar a compatibilidade de ambas.
%
%\section{Definição de um conjunto mínimo de facetas corporativas}
%
%%\textbf{Técnicas usadas.}
%O estabelecimento de um conjunto mínimo de facetas, que possa atender adequadamente as necessidades de diferentes empresas, requer uma análise de domínio. A partir das duas coleções citadas na seção anterior, a análise de domínio adota as técnicas de análise de assuntos e de análise facetada. Ambas as técnicas servem ao propósito de descobrir características dos documentos corporativos e propor um conjunto de facetas comuns a ambas as coleções.
%
%%\textbf{Motivação da adoção da análise facetada para executar a análise de domínio.}
%A análise facetada permite ampliar facilmente esse conjunto de características, a\-co\-mo\-dan\-do-as em esquemas classificatórios hospitaleiros. Isso é útil para permitir que um esquema classificatório mais generalista seja personalizado para uma empresa específica, ou para uma unidade organizacional específica. Por outro lado, um conjunto genérico e realmente expressivo de características é útil para a interoperabilidade entre sistemas de informação, para o intercâmbio de informação corporativa, e principalmente para o projeto mais eficiente de sistemas de recuperação de informação corporativa.%Embora a organização facetada permita facilmente ampliar este conjunto, espera-se que toda a avaliação se concentre em um núcleo pequeno de características, mas de diferentes facetas.
%
%%\textbf{Onde encontrar o conjunto de facetas.}
%Os procedimentos metodológicos do processo de análise de domínio, através da análise de assunto e da análise facetada, são descritos no capítulo \ref{analiseDominio}.
%
%%Facetas correspondentes a nomes podem ser facilmente estendidas para nomes de pessoas (funcionários, clientes e colaboradores) e de outras empresas (clientes, fornecedores e subsidiárias). Facetas correspondentes a locais incluem endereços dos diversos atores socio-técnicos existentes, os quais também possuem nomes. Finalmente, facetas relacionadas a tempo referem-se a datas de nascimento, fundação ou inauguração, ocorrência de atendimentos e de publicação de documentos, dentre outras. Outras facetas podem depender de uma combinação ou sobreposição de duas ou três das facetas citadas anteriormente, como é o caso daquelas relacionadas a processos.
%
%%\textbf{Limitação do escopo da análise de domínio.}
%%A análise de domínio empreendida corresponde a uma tentativa preliminar. O conjunto de facetas resulta de uma análise de domínio preliminar, inicialmente partindo da revisão da literatura e dos dois exemplares do domínio, citados na seção anterior. O estudo de exemplares do domínio serve para que características da linguagem, adotada por diferentes atores sociais, sejam explicitadas. 
%
%%Os trabalhos encontrados na literatura indicam que muita atenção é dada às entidades que se encontram explícitas nas mensagens entre atores sociais que se comunicam por meio de linguagens técnicas específicas, tais como financeira, jurídica, de informática, de engenharia ou de \textit{marketing}. Mesmo dentro da mesma empresa, entidades são tratadas sob perspectivas muito diferentes dentro das linguagens técnicas específicas, apresentando facetas muito particulares de interesse. Pela análise facetada, são descobertas as facetas dessas entidades que permeiam a maioria ou totalidade das linguagens técnicas específicas. Essas facetas têm o potencial de representar as expressões de busca dos usuários corporativos e de representar minimamente as entidades que são citadas em documentos corporativos, compondo índices de sistemas de recuperação de informação assim como também foi ilustrado na figura \ref{fig:metodologia}. 
%
%\section{Prototipação de um sistema de recuperação de informação corporativa}
%
%%\textbf{Experimento empírico via protótipo.}
%A prototipação baseada em \textit{software} é descrita na figura \ref{fig:metodologia} apenas como protótipo. O protótipo de um sistema de recuperação de informação, implementado em linguagem Java e usando a biblioteca Lucene, executa as funções de indexar e recuperar documentos apenas da coleção pública para realizar um experimento empírico. O protótipo é documentado no capítulo \ref{prototipo}.
%
%%\textbf{Metodologia de prototipação.}
%A metodologia de prototipação se baseia no trabalho de \citeonline{anastacio09}, pela adoção de mecanismos comuns de coleta, classificação, indexação, busca e recuperação de documentos e sua adaptação para necessidades especiais. No contexto desta tese, os requisitos do protótipo são: organização facetada, tratamento espaço-temporal, e reconhecimento de entidades sociais, espaciais e temporais. Pela implementação desses requisitos, um sistema de recuperação de informação comum torna-se um sistema de recuperação de informação corporativa e facetada.
%
%%\textbf{Requisitos não-funcionais e funcionais.}
%Os requisitos não-funcionais do protótipo incluem o tratamento geográfico através da indexação espacial e da implementação de um \textit{gazetteer}; o tratamento temporal através da indexação de indicadores de tempo; e a federação dos repositórios corporativos através da extração de texto dos documentos coletados e a geração de um índice centralizado. Os requisitos funcionais do protótipo corporativo e facetado incluem a indexação, a interface de busca com o usuário, a busca em lote para avaliação de várias expressões de busca em conjunto, e a recuperação de informação.
%
%%\textbf{Requisitos não considerados.}
%O controle de acesso a documentos, um importante requisito não-funcional dos sistemas de recuperação de informação corporativa, não é considerado pela natureza da coleção de documentos da CSIRO, constituída apenas por dados acessíveis a todos os usuários. Os efeitos dessa omissão não afetam o cenário de avaliação proposto pela trilha \textit{Enterprise} da \textit{Text Retrieval Conference}. %e nenhum trabalho da literatura parece incorporar esse requisito.
%
%%\textbf{Método experimental se aplica apenas à coleção pública.}
%Essa avaliação baseada em prototipação constitui um importante método de validação de resultados desta tese. Porém, sua utilidade é limitada apenas a uma das duas coleções de documentos adotadas, como se observa na figura \ref{fig:metodologia}. A validação de resultados da coleção particular se dá através dos usuários de informação e não beneficia-se diretamente desse método experimental.
%
%
%%Quais os requisitos mínimos devem ser respeitados para coleta, \textit{parsing}, reconhecimento e desambiguação de facetas, indexação, busca, processamento de consulta, ranking e apresentação?
%
%%iv) Projetar e implementar componentes do SRI corporativo para oferecer suporte a organização facetada da informação, a saber: espaço, tempo, remetente, destinatário, assunto e outras facetas já estudadas por \citet{pontesLima2012, maculan2011, anastacio09} e por \citet{cardosoSantos08}.
%
%%Geo está ok. Quais outras facetas? Como implementar? O que será preciso estudar? Há certeza sobre todas as facetas disponíveis?
%
%\section{Avaliação e validação}
%
%%\textbf{2 cenários de avaliação.}
%Dois cenários de avaliação são adotados e ilustrados na figura \ref{fig:metodologia}. O primeiro cenário de avaliação refere-se à avaliação experimental da trilha \textit{Enterprise} da \textit{Text Retrieval Conference}, pela comparação dos resultados desta tese com aqueles de outros trabalhos disponíveis na literatura. Os principais resultados experimentais foram publicados no ano de 2008, quando a trilha foi extinta. O segundo, por sua vez, refere-se exclusivamente à coleção particular e baseia-se em seus usuários. Esse cenário de avaliação usa as expressões de busca elaboradas pelos usuários de informação para avaliar se as características da informação corporativa são reconhecidas por seus usuários. 
%
%%\textbf{Onde encontrar a avaliação.}
%Os métodos de avaliação e validação são detalhados no capítulo \ref{prototipo}, juntamente com os resultados da avaliação, sua análise e discussões.
%
%%Usuários da segunda coleção foram empregados no projeto e na execução da segunda avaliação.
%
%%Na coleção privada a avaliação será projetada exclusivamente para este trabalho, por reconhecimento prévio manual de relevância sobre um número reduzido de necessidades informacionais. Espera-se verificar diferentes métricas de desempenho para o processamento de consulta sobre busca em texto completo, busca em metadados, busca em taxonomia facetada classificada intelectualmente e em taxonomia facetada classificada automaticamente a partir de texto completo. A taxonomia facetada refere-se a uma taxonomia criada especificamente para as coleções após a leitura dos documentos. 
%
%\section{Anotação da coleção particular}
%
%%\textbf{Caracterização da coleção particular.}
%A coleção pública de documentos, ou coleção de referência, conta com anotações feitas por especialistas. Tais anotações referem-se a caracterização da coleção, listagem de documentos e pessoas relevantes, e de buscas mais frequentes. A coleção particular, usada primeiramente nesta tese, necessita desse tipo de caracterização para ser usada em futuros trabalhos sobre organização da informação corporativa.
%
%%\textbf{Importância da anotação.}
%Como demonstrado na figura \ref{fig:metodologia}, a atividade de anotação dessa coleção é realizada a partir da documentação do experimento sobre a coleção particular, constituindo o último produto desta tese. Após a disponibilização de uma réplica da coleção particular, por seus proprietários, a coleção e a sua anotação garantem que os resultados desta tese podem ser repetidos e validados em trabalhos futuros.
%
%%Portanto, a coleção particular torna-se disponível para a comunidade científica apenas após o encerramento deste trabalho, quando os detentores de direitos sobre a coleção particular passam a conhecer os assuntos descobertos em seus próprios dados.
%
%
%%\textbf{Disponibilização da }
%%Uma réplica da coleção particular deve tornar-se pública após uma análise cuidadosa de aspectos relacionados à propriedade intelectual, propriedade industrial e dados sensíveis, sendo fundamental para trabalhos futuros.
%
%
%%Amazon Turk será usado? O que será manual e o que será automático? Qual o \% de acerto? Precisão! Quanto de cobertura? Revocação! Quanto à busca e ao processamento de consulta, o que combinar? Como? Comparar taxonomia facetada vs taxonomia dinâmica vs fulltext estatístico vs fulltext multifacetado. Qual critério? Precisão? Revocação? Tempo? Side by side? Tarefas trec?
%
%%Tarefas TREC serão comparadas pelos resultados disponibilizados na literatura. Outras tarefas serão comparadas por trabalhos prévios sobre taxonomia facetada de \citet{maculan2011}, para a BDTD da ECI/UFMG, sobre 41 documentos, e taxonomias dinâmicas de \citet{pontesLima2012} para a mesma BDTD.
%
%
%%A metodologia proposta para esta pesquisa é principalmente experimental, para qual é necessário: i) coletar e reunir uma massa de dados convencional da Web, com diferentes tipos de documentos, idiomas e gêneros linguísticos; ii) projetar e implementar um arcabouço de \textit{software} que indexe a massa de dados de páginas Web usando uma estrutura de dados geométricos, e faça o reconhecimento de topônimos através de um \textit{gazetteer} e das referências geográficas da própria massa de dados indexada; iii) projetar o primeiro cenário de avaliação baseado naqueles existentes no \textit{framework} GeoCLEF, utilizando-se da massa de dados do \textit{framework}; iv) projetar o segundo cenário de avaliação baseado na massa de dados construída para este trabalho; v) estudar e comparar o desempenho das técnicas propostas com os indicadores apresentados no GeoCLEF; vi) produzir anotações semiautomáticas na massa de dados própria como proposta para avaliação de técnicas de RIG em conjuntos heterogêneos de documentos quanto ao tipo, idioma e gênero linguístico.
%
%%
%%
%%\section{Planejamento das atividades}
%%\label{planejamento}
%%
%%O cronograma ilustrado na figura \ref{fig:cronograma} apresenta as próximas atividades necessárias para a implementação deste projeto. O cronograma apresenta apenas as atividades entre fevereiro de 2013 e fevereiro de 2014. O início do período corresponde exatamente ao início da elaboração deste texto, embora procedimentos metodológicos importantes tenham acontecido antes. As atividades estão correlacionadas a capítulos da tese, alguns já presentes neste projeto.
%%
%%\begin{figure}
%%
%%	\caption{\label{fig:cronograma}Cronograma}
%%
%%	\centering
%%		\includegraphics[width=1.00\textwidth]{fig/cronograma.jpg}
%%	\legend{Fonte: elaborada pelo autor}
%%\end{figure}
%%
%
%
%%\begin{center}
%%\setlength{\tabcolsep}{0.0mm}
%%\renewcommand{\arraystretch}{1.25}
%%\begin{tabular}{|l|p{\constlen}||p{\constlen}|p{\constlen}||p{\constlen}|p{\constlen}||p{\constlen}|p{\constlen}||p{\constlen}|}
%%\cline{2-9}
%%\multicolumn{1}{c}{} & \multicolumn{1}{|c||}{\textbf{2010}} & \multicolumn{2}{|c||}{\textbf{2011}} & \multicolumn{2}{|c||}{\textbf{2012}} & \multicolumn{2}{|c||}{\textbf{2013}} & \multicolumn{1}{|c|}{\textbf{2014}} \\ \hline
%%~{\textbf{Tarefas}} &\mk{\textbf{2S}}&\mk{\textbf{1S}}&\mk{\textbf{2S}}&\mk{\textbf{1S}}&\mk{\textbf{2S}}&\mk{\textbf{1S}}&\mk{\textbf{2S}}&\mk{\textbf{1S}} \\ \hline
%%~Disciplinas &\pl{3} & & & & & & & \\ \hline
%%~Estágio de docência & & & & & \pl{2} & & & \\ \hline
%%~Ajustes de reopção  & & & & & \pl{2} & & & \\ \hline
%%~Projeto de Tese     & & & & \pl{1} & & & & \\ \hline
%%~Execução da Pesquisa& & & & \pl{4} & & & & \\ \hline
%%~Escrita da Tese{~} & & & & & & \pl{2} & & \\ \hline
%%~Defesa da Tese & & & & & & & & \pl{1}  \\ \hline
%%~Produção de Artigos & & & & & \pl{1} & & \pl{2} & \\ \hline
%%~Período no DCC & \pl{3} & & & & & & & \\ \hline
%%~Período na ECI & & & & \pl{5} & & & & \\ \hline
%%\end{tabular}
%%\end{center}
%
%
%
%
%
